\documentclass{article}
\usepackage[utf8]{inputenc}

\title{400b HW3}
\author{Emily Walla}
\date{February 2020}

\begin{document}

\maketitle
\section{Galactic Masses Table}

\begin{tabular}{lllllll}

  Galaxy Name &  Halo (M\textsubscript{\(\odot\)}) &  Disk (M\textsubscript{\(\odot\)}) &  Bulge (M\textsubscript{\(\odot\)}) &  Total (M\textsubscript{\(\odot\)}) &     f\_bar \\

   Milky Way &       1.975e+14 &       7.500e+12 &        1.001e+12 &   2.060e+14 &  0.041 \\
         M31 &       1.921e+14 &       1.200e+13 &        1.905e+12 &   2.060e+14 &  0.068 \\
         M33 &       1.866e+13 &       9.300e+11 &        0.000e+00 &   1.959e+13 &  0.047 \\
 Local Group &       4.082e+14 &       2.043e+13 &        2.905e+12 &   4.316e+14 &  0.054\\
\end{tabular}

\section{Q1}

Total Galaxy Mass:
\\
\begin{tabular}{ll}

Milky Way &     2.060e+14 M\textsubscript{\(\odot\)} \\
M31 &     2.060e+14 M\textsubscript{\(\odot\)}


\end{tabular}
\\
\\
The total masses of the Milky Way and M31 are the same.  Both galaxies' masses are dominated by their halo mass.

\section{Q2}

Stellar Mass:
\\
\begin{tabular}{ll}

Milky Way &     8.501e+12 M\textsubscript{\(\odot\)} \\
M31 &     3.105e+12 M\textsubscript{\(\odot\)}

\end{tabular}
\\
\\
Milky Way's stellar mass is 2.738*M31's stellar mass.
The Milky Way's stellar mass is quite a bit bigger than M31's stellar mass, so I would expect the Milky Way to be more luminous

\section{Q3}
Dark Matter:
\\
\begin{tabular}{ll}

Milky Way &     1.975e+14 M\textsubscript{\(\odot\)} \\
M31 &     1.921e+14 M\textsubscript{\(\odot\)}

\end{tabular}
\\
\\
Milky way's dark matter mass is 1.028*M31's dark matter mass. I'm not surprised that they have similar amounts of dark matter, because their stellar masses are still within order of magnitude of each other, and I believe they have roughly similar radii.

\section{Q4}
Baryon Fraction:
\\
\begin{tabular}{lll}
Galaxy &     f-bar &    percent \\
Milky Way &     0.041 &    4.1 \\
M31 &     0.068 &     6.8\\
M33 &     0.054 &     5.4

\end{tabular}
\\
\\
There's a heck of a lot more gas out there in the universe than is locked up in galaxies, and some galaxies may have a higher Baryon fraction. These galaxies are spiral galaxies -- I don't know enough to say for sure, but this question makes me think that maybe spiral galaxies just have lower baryon fractions than other types, like irregular and elliptical galaxies. So the intergalactic gas and the type of galaxy seen here might be why the baryon fraction of the universe differs from what we see in these galaxies.
\end{document}
